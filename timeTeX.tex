% TeX Machine: LuaLaTeX
% Author: Fabian Krautgasser

\documentclass[landscape, a4]{article}

% encoding, language and font
\usepackage[utf8]{luainputenc}
\usepackage[english, ngerman]{babel}
\usepackage[T1]{fontenc}
\usepackage{AnonymousPro}  
\renewcommand*\familydefault{\ttdefault} %% Only if the base font of the document is to be typewriter style

% packages for drawing and the size calculations
\usepackage{tikz}
\usepackage{pgfmath}

% setting the page margins to 1cm
\usepackage[margin=1cm]{geometry}

\begin{document}
	\pagestyle{empty}

	% The part below is not needed, since all distances are set by hand. It's kept for nostalgic reasons.
	%%% INITIAL CALCULATIONS %%%
	
	% finding the longest day as in the day with the longest name in pts. 
	\newlength{\tmpday}
	\pgfmathsetlengthmacro{\longestday}{0pt}
	\foreach \day in {"Monday", "Tuesday", "Wednesday", "Thursday", "Friday"}
	{
		\pgfmathparse{width(\day)}
		\pgfmathsetlength{\tmpday}{\pgfmathresult}
		\pgfmathmax{\longestday}{\tmpday}
		\global \let \longestday \pgfmathresult
	}
	
	% adding 1.5cm to the length for the spacing between the days
	\pgfmathadd{\longestday}{1.5cm}
	\pgfmathdivide{\pgfmathresult}{1cm}  % convert to cm
	\global \let \longestday \pgfmathresult
	

	%%%% Definitions %%%%
	\newcommand{\mainTitle}{WS 17/18}
	\newcommand{\timetex}{T \raisebox{-3pt}{\kern -7pt I}\kern -1.5pt M \raisebox{-1pt}{\kern -6pt \small E} \kern -6pt \TeX\ }



	% Main frame
	\begin{figure}[h]
	\centering

	\begin{tikzpicture}[%
									   % style for classes
									   class/.style = {draw=none, 
															fill=white!20, 
															rectangle, 
															align=center, 
															anchor=north, 
															text width=2.5cm, 
															minimum height=1.75cm},
										% style for the classes' location (default: grey, scriptsize text)				 
										loc/.style = {draw=none, 
															anchor=south, 
															text width=2.5cm, 
															yshift=-0.1cm, 
															text=gray,
															align=right,
															font=\scriptsize},
								% style for parallel classes; 		
								para-left/.style = {anchor=north east,
															text width=1.1cm,
															xshift=-0.025cm},
							  para-right/.style = {anchor=north west,
														   text width=1.1cm,
														   xshift=0.025cm},
								para-loc/.style = {text width=1.1cm},
								timeline/.style = {color=red, very thick},
							  separator/.style = {color=black, dotted, thin}]
								
	
		%%%% BACKGROUND %%%%
		\draw [fill=gray!20, draw=none] (-3,2) rectangle (15.5,-15);
		
		%%%% TITLE %%%%
		\node at (.23\textwidth, 1cm) (title) {\textbf{\mainTitle}};
		
		%%%%% version with date %%%%
		\node [anchor=east] at (15,-14.5) (version) {v[\today]};
		
		%%%% LOGO %%%%
		\node[anchor=east] at (15.2, 1.5) (logo) {\timetex};
				
		%%%% RED LINES FOR TIME %%%% 
		\draw [-|, timeline] (-2, -13.25) -- (-2,-14) ; 
		\draw [-|, timeline] (12.75, 0) -- (14,0) ; 
		\draw [timeline] (-2,0) -- (-2, -0.5);
		
			%%% y ticks %%%
			\foreach \count in {0,1,2,...,52}
				{
				\draw (-1.8, -0.25*\count-0.75) -- (-1.6, -0.25*\count-0.75); 
				}
			\foreach \count in {1,2,...,13}
				{
				\draw [thick] (-1.8, -1*\count) -- (-1.5, -1*\count);
				}
			
			%%% vertical bars with hourly ticks %%%
			\foreach \x in {1.5, 4.5, ..., 10.5}
				{
				\draw [separator] (\x, -0.75) -- (\x, -14);	
				\foreach \y in {-1,-2,...,-13}  %ticks
					{
					\draw [separator] (\x - 0.1, \y) -- (\x + 0.1, \y);	
					}
				}
			
			%%% days %%%
			\foreach [count=\x from 0] \day in {Monday, Tuesday, Wednesday, Thursday, Friday}
				{
				\node at (\x*3,0) (\day) {\day};
				\draw [timeline] (\x*3 -2. ,0) -- (\day.west);
				}
		
		    %%% hours %%%
		    \foreach [count=\y from 1]  \hour in {8,9,...,20}
			    {
			    \node at (-2, -1*\y) (\hour) {\hour};
			    \draw [timeline] (-2, -1*\y +0.75) -- (\hour.north);
			    } 
			
		% declare math variables
		\pgfmathsetmacro{\xloc}{0}  % for setting the space between days
		\pgfmathsetmacro{\sloti}{-2.25} % defining the time slots
		\pgfmathsetmacro{\slotii}{-4.25} % defining the time slots
		\pgfmathsetmacro{\slotiii}{-7.25} % defining the time slots
		\pgfmathsetmacro{\slotiv}{-9.25} % defining the time slots
		
		%%%% MONDAY %%%%
		\node[class, minimum height=1cm] at (\xloc, -2) (gmeet) {Group meeting};
		\node[loc] at (gmeet.south) {Meeting Room I};		
		
		%%%% TUESDAY %%%%
		\pgfmathsetmacro{\xloc}{\xloc + 3}  
		\node[class] at (\xloc, \sloti) (dtop1) {Differential- topology I};
		\node[loc] at (dtop1.south) {Seminar room 2};
		\node[class] at (\xloc, \slotii) (statphys1) {Theoretical Statistical Physics};
		\node[loc] at (statphys1.south) {Room XIV};
		
		%%%% WEDNESDAY %%%%
		\pgfmathsetmacro{\xloc}{\xloc + 3}
		\node[class, minimum height=1.5cm] at (\xloc, -12.5) (yoga1) {Yoga Class};
		\node[loc] at (yoga1.south) {Gym};
		

		%%%% THURSDAY %%%%
		\pgfmathsetmacro{\xloc}{\xloc + 3}
		\node[class] at (\xloc, \sloti) (dtop2) {Differential- topology};
		\node[loc] at (dtop2.south) {Seminar room 2};
		\node[class] at (\xloc, \slotii) (statphys2) {Theoretical Statistical Physics};
		\node[loc] at (statphys2.south) {Room XIV};
		\node[class] at (\xloc, \slotiv) (astrobio) {Astrobiology \& Astrobio- physics};
		\node[loc] at (astrobio.south) {Observatory SR};
		
		%%%% FRIDAY %%%%
		\pgfmathsetmacro{\xloc}{\xloc + 3}
		\node[class] at (\xloc, \slotiii) (statphystut) {Tensor Calculus};
	
	\end{tikzpicture}
	\end{figure}

\end{document}
